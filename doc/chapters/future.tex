
%%%%%%%%%%%%%%%%%%%%%%%%%%%%%%%%%%%%%%%%%%%%%%%%%%%%%%%%%%%%%%%%%%%%%%%%
\chapter{Conclusion and Future Work}
\label{chap:future}
%%%%%%%%%%%%%%%%%%%%%%%%%%%%%%%%%%%%%%%%%%%%%%%%%%%%%%%%%%%%%%%%%%%%%%%%

% We examined intersection types and union types in various settings in this thesis.
% This involves a type-sound calculus with first-class intersection types and union types.
% First calculus lacks determinism. Second calculus is developed without
% merge operator. This calculus is type-sound and deterministic. Merge operator
% and type-based switch expression induce non-determinism together. A parallel work
% to solve non-coherence problem in the presence of merge operator is
% disjoint intersection types \cite{}.

\section{Conclusion}

The integration of the intersection and union
types is known to be non-trivial in theory.
This thesis examines the integration of
intersection and union types in
various settings. \Cref{chap:switch} (\name)
discusses the deterministic elimination of
union types with a type-based switch construct.
The disjointness plays the essential role
in making such a construct deterministic.
The disjointness ensures that no two
types in alternative branches overlap i.e.
do not share a common subtype. Therefore,
the type of the scrutinee matches with one branch
at the most. Later sections in \Cref{chap:switch}
further enrich \name with advance features
including intersection types, nominal types,
subtyping distributivity and disjoint polymorphism.
\Cref{chap:disjointness} discusses an
expressive version of the disjointness algorithm
with disjoint polymorphism.
All of the calculi discussed in \Cref{chap:switch}
and \Cref{chap:disjointness} preserve the
standard properties of type-safety and determinism.

\Cref{chap:merge} studies a calculus (\namems) with
an elimination construct for the union types
and an introductory construct for the intersection types,
the so called merge operator.
\namems is type-safe but lacks
determinism. We also prove the completeness of \namems
with respect to \cite{dunfield2014elaborating}.
\Cref{chap:duo} discusses \nameduo,
a novel technique studied to unify the subtyping rules
for the dual features such as intersection
types and union types. \nameduo comes with
certain benefits of reduced subtyping rules,
easier proofs and metatheory,
and extra features such as lower bounded
quantification. We study \nameduo with various
calculi to show that it is a practically
applicable technique.

% We explored the duality of subtyping in second part of this thesis.
% We show that duality comes with certain benefits including
% shorter specification, extra features and easier proof techniques.
% We also show that \nameduo is a practically applicable technique by studying it
% with a number of calculi.

% One future line of work is studying merge operator and switch together and make
% them deterministic. We have discussed a few techniques to deal with the
% non-determinitic problem in section \ref{}.
% \baber{talk briefly about the techniques.}


\section{Future Work}


\subsection{Determinism for \namems}
\label{future:proposals}

Future work includes making the calculus
discussed in \Cref{chap:merge} (\namems) deterministic.
We discuss a few proposals to make \namems
deterministic.
Approaches include restricting subsumption
rule to prohibit certain ambiguous upcasts,
first-match semantics, and parallel application.
%of the branches that pass the type test.

%---------------------------------------------------%%

%%%%%%%%% Remediations for Non-determinism %%%%%%%%%%%

%---------------------------------------------------%%



% \section{Proposals for deterministic \namems}
% \label{future:proposals}

%\baber{Remediation should be a part of future work?}

\subsubsection{Proposal 1: restricting ambiguous upcasts}

The meticulous observation concludes that the
origin of non-determinism is primarily due to multiple upcast paths.
For example, $[[1,,true]]$ can follow two paths to upcast
to $[[Int\/Bool]]$. One is via $[[Int]]$ and the other is via $[[Bool]]$.
Our first proposal to deal with non-determinism is to reject
the programs when there are multiple upcast paths involved.
In this approach the following program will be rejected:

\begin{lstlisting}[language=Scala]
  Bool isInt (x : Int | Bool) = switch (x)
                                  (x:Int)  -> true
                                  (y:Bool) -> false

  isInt(1,,true) //rejected
\end{lstlisting}

\noindent We propose restricting subsumption rule with an extra condition
so that it allows only unambiguous upcasts along with disjointness
in merges and switches. Such a calculus will have three measures
for determinism. One is the disjointness in switches, another is the
disjointness in merges, and finally an extra restriction in
subsumption rule to reject ambiguous upcasts. The revised subsumption
rule is shown next:

%\baber{discuss restricted subsumption rule and unambuguos relation.}

\begin{mathpar}
  \inferrule*[Right=typ-sub-res]
    { [[G |- e: A]] \and
      [[A <: B]] \and
      \mathcolorbox{lightgray}{[[A]] <_u [[B]]}
    }
    {[[G |- e:B]]}
\end{mathpar}

\noindent Where, $\mathcolorbox{lightgray}{[[A]] <_u [[B]]}$
indicates type $[[A]]$
is unambiguous to type $[[B]]$. Meaning that there is at most one path
to upcast from $[[A]]$ to $[[B]]$.
Ideally, this relation allows upcasting $[[1:Int]]$ to $[[1:Int\/Bool]]$.
But it does not allow upcasting $[[1,,true : Int/\Bool]]$
to $[[1,,true : Int \/ Bool]]$. This is because $[[1:Int]]$
upcasts to $[[1:Int\/Bool]]$ via only one path i.e.:

\begin{mathpar}
  \inferrule*[Right=\rref*{typ-sub}]
    { [[G |- 1: Int]] \and
      [[Int <: Int\/Bool]]
    }
    {[[G |- 1:Int\/Bool]]}
\end{mathpar}\noindent

\noindent Whereas, $[[1,,true]]$ has two paths to upcast to $[[Int\/Bool]]$,
one via $[[Int]]$ and the other via $[[Bool]]$.

\begin{itemize}
  \item[\textbf{1)}] \textbf{First path:} Upcast of $[[1,,true]]$ to $[[Int\/Bool]]$ via $[[Int]]$:
\end{itemize}

\begin{mathpar}
  \inferrule* [Left=\rref*{typ-sub}]
    { \inferrule* [Left=\rref*{typ-merga}]
        { }
        {[[1,,true : Int/\Bool]]} \\
      \inferrule*  [Right=\rref*{s-orb}]
        { \inferrule* [Right=\rref*{s-andb}]
          {
            [[Int <: Int]]
          }
          {[[Int/\Bool <: Int]]}
        }
        {[[Int/\Bool <: Int \/ Bool]]}
    }
    {[[G |- 1,,true : Int\/Bool]]}
\end{mathpar}


\begin{itemize}
  \item[\textbf{2)}] \textbf{Second path:} Upcast of $[[1,,true]]$ to $[[Int\/Bool]]$ via $[[Bool]]$:
\end{itemize}

\begin{mathpar}
  \inferrule* [Left=\rref*{typ-sub}]
    { \inferrule* [Left=\rref*{typ-merga}]
        { }
        {[[1,,true : Int/\Bool]]} \\
      \inferrule* [Right=\rref*{s-orc}]
        { \inferrule* [Right=\rref*{s-andc}]
          {
            [[Bool <: Bool]]
          }
          {[[Int/\Bool <: Bool]]}
        }
        {[[Int/\Bool <: Int \/ Bool]]}
    }
    {[[G |- 1,,true : Int\/Bool]]}
\end{mathpar}

\noindent Therefore the upcast of $[[1,,true]]$ to $[[Int\/Bool]]$
is rejected with
the restricted subsumption rule.
The essence of restricted subsumption rule is to reject the
programs that may be a cause of non-determinism in switch
expression.

% \begin{center}
% \drule[]{typ-sub}
% \end{center}

%\baber{discuss challenges of this approach, specially with functions.}


\paragraph{Challenges with unambiguous relation.}
Restricting multiple upcasts with unambiguous relation
may assist to make the calculus deterministic but
unambiguous relation itself is non-trivial in metatheory.
Specifically, it becomes challenging with multiple
type annotations with functions.
For example, $[[\x.e:Int -> Int/\Bool : Int -> Int : Int -> Int\/Bool]]$
is deterministic as long as we carry the middle type i.e $[[Int -> Int]]$.
As soon as we drop the middle type, application of such lambdas may
become non-deterministic.
The contravariance of function input type adds further
complexities.
Possible remedies in such situations is to carry a list of
type annotations with functions.

% \paragraph{Carry only one type annotation with lambdas and drop type annotations.}

% $[[\x.e:A->B]]$ instead if $[[\x.e:A1->B1:A2->B2]]$.
% Problem occurs when type casting lambda with two
% annotations.

\subsubsection{Proposal 2: first-match semantics}

A natural approach to select a particular branch of a switch
expression is to follow the first-match semantics,
meaning that first branch that matches the type
of scrutinee will be selected.
In this remedy we propose to keep disjointness in merges but
eliminate disjointness from switches. Merges will be deterministic
as in disjoint intersection types \cite{oliveira2016disjoint}.
Whereas, the switches will follow the first-match semantics.
% Meaning that instead of randomly falling into a branch, first matched branch
% with the scrutinee will be chosen.
% First match semantics is implemented in many practical
% programming languages \cite{} \baber{citations}.

\noindent For example, with this approach, first branch will be
selected in the following code. This is because $[[1,,true]]$
can be considered as a value of type $[[Int]]$ and the type
of scrutinee i.e $[[Int/\Bool]]$ matches (is a subtype of) 
$[[Int]]$.

\begin{lstlisting}[language=Scala]
  Bool isInt (x : Int | Bool) = switch (x)
                                  (x:Int)  -> true
                                  (y:Bool) -> false

  isInt (1,,true)
\end{lstlisting}

\noindent However, first-match semantics selects first 
branch in following code as well because of the same
reason but with type $[[Bool]]$ in first branch.

\begin{lstlisting}[language=Scala]
  Bool isInt (x : Int | Bool) = switch (x)
                                  (x:Bool)  -> false
                                  (y:Int)   -> true

  isInt (1,,true)
\end{lstlisting}

\noindent The semantics of the
program may change by reordering the branches in this approach.
The first code snippet returns $[[true]]$, whereas the second
code snippet returns $[[false]]$.
% $[[1,,true]]$
% can be considered as value of type $[[Bool]]$, so as per first match
% semantics it falls in the first branch i.e $[[Bool]]$ and returns
% false. Whereas, $[[1,,true]]$ is also $[[Int]]$.

% However, analyzing closely
% the issue in our setting, non-determinism only occurs when a merge is
% passed to the switch expression which may nullify the decisive power of
% switch. By closely looking into the exambel above:

% \begin{lstlisting}
% 1,,True : Int & Bool : Int : Int | Bool

% 1,,True : Int & Bool : Bool : Int | Bool
% \end{lstlisting}

% There are two paths to go from $[[Int /\ Bool]]$ to $[[Int \/ Bool]]$.
% One path is via Int and other is via Bool.


\subsubsection{Proposal 3: parallel application}
Another proposal is to adopt the parallel application.
In this approach, the code in all of the branches will be
executed to which the type of scrutinee matches. Final result
will be a merge of the values from all of the (match) branches.
For example, the result of the following
program will be $[[(false,,true)]]$.

\begin{lstlisting}[language=Scala]
  Bool isInt (x : Int | Bool) = switch (x)
                                  (x:Bool)  -> false
                                  (y:Int)   -> true

  isInt (1,,true)
\end{lstlisting}

\paragraph*{Challenge with parallel application.}
However, on of the challenges in naive implementation
of this approach is that it may generate ill-typed programs
as in the above example. The return value
$[[(false,,true)]]$ is an ill-typed program
in disjoint intersection types.
This is because $[[(false,,true)]]$ is of type
$[[Bool/\Bool]]$ and $[[Bool]]$ is not disjoint
to $[[Bool]]$. If we allow type-checking such
programs, the overall calculus will still be
non-deterministic due to the ambiguous merge operator.
We further propose two approaches to deal with such an issue.

\paragraph{Disjoint return type of branches.}
One approach to solve this problem is to employ disjointness
in the return type of alternative branches.
For example, the following problematic program will
be rejected in this case:

\begin{lstlisting}[language=Scala]
  Bool isInt (x : Int | Bool) = switch (x)
                                  (x:Bool)  -> false
                                  (y:Int)   -> true

  isInt (1,,true)
\end{lstlisting}

\noindent This is because both of the branches return values
of type $[[Bool]]$ and $[[Bool]]$ is not disjoint
to $[[Bool]]$. Whereas, the following program will
be accepted:

\begin{lstlisting}[language=Scala]
  Int | Bool notSucc (x : Int | Bool) = switch (x)
                                          (x:Bool)  -> not x
                                          (y:Int)   -> succ x

  notSucc (1,,True)
\end{lstlisting}

\noindent Note that the first branch returns $[[Bool]]$ and
the second branch returns $[[Int]]$. Since $[[Int]]$
is disjoint to $[[Bool]]$, therefore, it is safe to accept
such programs.

\paragraph{Record type for explicit disjointness.}
The second approach is to use record types with distinct
labels as a return type of alternative branches to enforce disjointness.
The return type of a switch in this case will always be
sound. Therefore, the calculus will not generate
ill-typed programs at runtime.

\begin{lstlisting}[language=Scala]
  {l1:Bool, l2:Bool} isInt (x : Int | Bool) = switch (x)
                                          (x:Int)  -> {l1:true}
                                          (y:Bool) -> {l2:false}

  isInt (1,,true)
\end{lstlisting}

% \subsection{Remedy 6: Only one type annotation with functions}

% Yet another approach is to to have disjoint
% return type of branches.


\subsection{Multiple Interface Inheritance}

Multiple interface inheritance is a prominent feature
available in many modern programming languages including
Java and Scala.
This feature is essential for the extensibility and scalability
of software development. The calculi discussed in this thesis
do not support multiple interface inheritance.
Another line of future work is to allow multiple
interface inheritance in \name.

Specifically, \name with nominal types can further
be enriched to support multiple inheritance.
The lack of multiple interface inheritance is primarily
due to the fact that only the $[[Top]]$ type or a nominal
type can be declared as a supertype of another nominal type
in $[[PG]]$. For example, in a nominal type environment
with $\Delta = \mathsf{Person} [[<:]] [[Top]],
\mathsf{Robot} [[<:]] [[Top]]$, the following new
declaration is allowed:

\begin{center}
$\Delta = \mathsf{Person} [[<:]] [[Top]],
\mathsf{Robot} [[<:]] [[Top]],
\mathcolorbox{lightgray}{\mathsf{Student} [[<:]] \mathsf{Person}}$
\end{center}

\noindent We add a new nominal type named
\lstinline{Student} and extend it with \lstinline{Person}.
Notice that only the $[[Top]]$ type or a nominal type
is declared as a parent type in $[[PG]]$.
An attempt to declare multiple parents of a nominal
type is rejected. For example, the following extension
of $[[PG]]$ is not allowed:

\begin{center}
$\Delta = \mathsf{Person} [[<:]] [[Top]],
\mathsf{Robot} [[<:]] [[Top]],
\mathcolorbox{lightgray}{\mathsf{Hybrid} [[<:]] \mathsf{Person}},
\mathcolorbox{lightgray}{\mathsf{Hybrid} [[<:]] \mathsf{Robot}}$
\end{center}

This restricts multiple inheritance
due to the fact that only one type can be
declared as a parent type
of another nominal type. It is essential for the
multiple interface inheritance that a nominal type
may define multiple parent types.
Multiple inheritance can be achieved by allowing
intersection types to be a supertype of nominal types
in $[[PG]]$. Specifically, allowing intersections
as parent types in $[[PG]]$ will allow the following
declaration, which naturally provides multiple
interface inheritance:

\begin{center}
$\Delta = \mathsf{Person} [[<:]] [[Top]],
\mathsf{Robot} [[<:]] [[Top]],
\mathcolorbox{lightgray}{\mathsf{Hybrid} [[<:]] \mathsf{Person}
[[/\]]\mathsf{Robot}}$
\end{center}




\subsection{Explicit Disjointness of Nominal Types}
The Ceylon language employs an \lstinline{of}
construct to explicitly declare the disjointness
of nominal types. We will elaborate this using the
following Ceylon code:

\begin{lstlisting}[language=Scala]
  abstract class Student() of PG | UG {}
  class PG() extends Student() {}
  class UG() extends Student() {}
\end{lstlisting}

\noindent The first line of the code creates a \lstinline{Student}
class and declares two subtypes of \lstinline{Student}, \lstinline{PG} 
and \lstinline{UG} using the \lstinline{of} construct.
This declarations marks \lstinline{PG} and \lstinline{UG} disjoint.
It is safe to use \lstinline{PG} and \lstinline{UG} in alternative
branches of a switch expressions. On the other hand, it is
prohibited to create a class that extends both \lstinline{PG} and
\lstinline{UG} to retain determinism. The class \lstinline{PG}
and the class \lstinline{UG} must later be defined in the code.
The formalization of such a construct is also an interesting line of
future work. In essence, the nominal type environment will be
revised to handle the \lstinline{of} construct. Each entry
in the revised $[[PG]]$ contain three components. The name of the
newly defined nominal type, its parent type(s), and its subtype(s).

\begin{center}
$\Delta = \mathsf{Student} [[<:]] [[Top]] ~ \mathcolorbox{lightgray}{* ~ \mathsf{[PG,UG]}},
\mathsf{PG} [[<:]] \mathsf{Student} ~ \mathcolorbox{lightgray}{* ~ \mathsf{[~]}},
\mathsf{UG} [[<:]] \mathsf{Student} ~ \mathcolorbox{lightgray}{* ~ \mathsf{[~]}}$
\end{center}

\subsection{Gradual Typing}

Gradual typing is another line of
interest in practical programming
languages with significant recent development.
Union types naturally provide
gradual typing in a restricted fashion
in such a way that actual type is among
the certain candidates from a union of types.
In gradual typing, on the other hand, no
information of the unknown type is statically available.
Inclusion of gradual typing with \name is another
line of future work with practical interest.
The naive addition of
gradual typing with disjoint switches may allow the following program
(where * denotes unknown type):

\begin{lstlisting}
    Bool isInteger (x : Int | *) {
    switch (x):
        Int  -> true
        *    -> false
    }
\end{lstlisting}

The type-based switch construct in this code snippet is exhaustive, but
how do we make sure that the two branches will not overlap?
This question gives birth to a particular research question
of redefining the disjointness in the presence of union types,
type-based switches, and the unknown type (*).
The unknown type makes the disjointness non-trivial because
of the unavailability of the static type information.

% Specifically, studying gradual typing with
% the disjointness proposed in \name is
% non-trivial. To make the matter worst,
% disjoint polymorphism adds new challenges
% with gradual typing.