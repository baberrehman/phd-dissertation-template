\chapter{Alternative Disjointness}
\label{appendix1}


We discuss another variant of the disjointness algorithm
that depends on so called Common Ordinary Subtypes
(COST) in this chapter. Recall that the disjointness
algorithm discussed in \Cref{sec:inter} is set-based (LOS).
The COST defined in this chapter is not set-based.
We also prove the equivalence of standard disjointness specifications
and a variant based on COST in this chapter.


%---------------------------------------------------------------------------%%

\begin{figure}[!h]
      \begin{small}
      \begin{center}
        \begin{tabular}{rcl}
          \toprule
          $[[A]], [[B]]$, $[[C]]$ & $\Coloneqq$ & $ [[Top]] \mid [[Bot]] \mid [[Int]] \mid [[A -> B]] \mid [[A \/ B]] \mid [[A /\ B]] \mid [[Null]] $ \\
          $[[e]]$ & $\Coloneqq$ & $ [[x]] \mid [[i]] \mid [[\x.e]] \mid [[e1 e2]] \mid [[switch e A e1 B e2]] \mid [[null]]$\\
          $[[v]]$ & $\Coloneqq$ & $ [[i]] \mid [[\x.e]] \mid [[null]] $ \\
          $[[G]]$ & $\Coloneqq$ & $ \cdot \mid [[G , x : A]] $ \\
          \bottomrule
        \end{tabular}

        \drules[s]{$[[A <: B]]$}{Subtyping}{top, int, bot, null, arrow, ora, orb, orc, anda, andb, andc}
        \drules[ord]{$[[ordinary A]]$}{Ordinary Types}{int,arrow,null}
        \drules[uo]{$[[UO A]]$}{Union Ordinary Types}{top,int,arrow,unit,and}
        \drules[usp]{$[[B <=u A u=> C]]$}{Union Splittable Types}{or,orandl,orandr}
      \end{center}
      \end{small}
  \caption{Syntax, subtyping, ordinary, union ordinary and union splittable types for \name with intersection types.}
  \label{fig:cost:syntax:sub}
\end{figure}

%---------------------------------------------------------------------------%%

\paragraph{Syntax and subtyping.}
Syntax and the subtyping stays the same as in
\name with intersection types and are shown in
\Cref{fig:cost:syntax:sub}. Types, expressions,
values, context and subtyping have already
been explained. Subtyping relation preserves
the standard properties of reflexivity and
transitivity.

\begin{lemma}[Subtyping Reflexivity]
  $[[A <: A]]$
\label{lemma:union:cost:sub:refl}
\end{lemma}

\begin{lemma}[Subtyping Transitivity]
  If $[[A <: B]]$ and $[[B <: C]]$ then $[[A <: C]]$
\label{lemma:union:cost:sub:trans}
\end{lemma}


%---------------------------------------------------------------------------%%

% \begin{figure}[t]
%   \begin{small}
%     \centering
%     %\drules[ord]{$[[ordinary A]]$}{Ordinary Types}{int,arrow,null}
%     \drules[uo]{$[[UO A]]$}{Union Ordinary Types}{top,int,arrow,unit,and}
%     \drules[usp]{$[[B <=u A u=> C]]$}{Union Splittable Types}{or,orandl,orandr}
%   \end{small}
%   \caption{Union ordinary, and union splittable types.}
%   \label{fig:cost:usp:uo}
% \end{figure}

%---------------------------------------------------------------------------%%


\section{Common Ordinary Subtypes (COST)}

COST play an integral role in the design of the disjointness algorithm
presented in this chapter. This section explains the COST in
detail by discussing the COST specifications as well as
the corresponding algorithm that computes COST.

\paragraph{COST Specifications.}
The specifications for the COST are shown in \Cref{cost:specs:def}.
It trivially states that two types $[[A]]$ and $[[B]]$
share a COST if there exist an ordinary type $[[C]]$
such that $[[C]]$ is subtype of $[[A]]$ and $[[B]]$.
The ordinary types are shown in the middle of \Cref{fig:cost:syntax:sub}.
$[[Int]]$, $[[A->B]]$, and $[[Null]]$ constitute ordinary types.


\begin{definition}[COST Specifications]
\label{cost:specs:def}
    A $\sqcap_s$ B $\Coloneqq$ $\exists$ C, $[[ordinary C]]$ and $[[C <: A]]$ and $[[C <: B]]$ \\
\end{definition}

\paragraph{Union ordinary and union splittable types.}
Union ordinary and union splittable types are shown at the bottom in
\Cref{fig:cost:syntax:sub}. These types have already been discussed
in \Cref{chap:disjointness}. Union ordinary types include
$[[Top]]$, $[[Int]]$,
$[[A->B]]$, $[[Null]]$ and an intersection of union
ordinary types. Any type that is not union ordinary is
union splittable.

%---------------------------------------------------------------------------%%

\begin{figure}[t]
  \begin{small}
    \centering
    \drules[cost]{$[[cost A B]]$}{Common Ordinary Subtypes}{top,ordl,ordr,int,null,arrow,orla,orlb,orra,orrb,andl,andr}
  \end{small}
  \caption{Common ordinary subtypes based on union splittable types for \name.}
  \label{fig:cost:cost}
\end{figure}

%---------------------------------------------------------------------------%%

\paragraph{COST Algorithm.}
The algorithm that computes COST is shown in \Cref{fig:cost:cost}.
In essence, the COST algorithm computes whether two given types
potentially share an ordinary subtype or not.
\Rref{cost-top} states that $[[Top]]$ shares an ordinary
subtype with $[[Top]]$. This is trivially true, such as $[[Int]]$
is a subtype of $[[Top]]$. \Rref{cost-ordl,cost-ordr} state that
$[[Top]]$ shares an ordinary subtype with all the ordinary types.
\Rref{cost-int,cost-null,cost-arrow} are the natural rules for
$[[Int]]$, $[[Null]]$, and $[[A->B]]$ respectively.

\Rref{cost-orla,cost-orlb,cost-orra,cost-orrb} deal with the
union splittable types. These rules collectively state that
if a type $[[B]]$ shares an ordinary subtype with a part ($[[A1]]$ or $[[A2]]$) 
of union splittable type ($[[A1 <=u A u=> A2]]$), then $[[B]]$
shares an ordinary subtype with $[[A]]$. This due to the fact that parts
($[[A1]]$ and $[[A2]]$) of a union splittable type
are subtypes of the original type ($[[A]]$).
\Rref{cost-andl,cost-andr}
deal with the intersection types. An intersection type 
$[[A1/\A2]]$ shares a ordinary subtype with $[[B]]$
if $[[A1]]$ shares an ordinary subtype with $[[B]]$,
$[[A2]]$ shares an ordinary subtype with $[[B]]$, and
$[[A1]]$ shares an ordinary subtype with $[[A2]]$.
A side condition of union ordinary $[[B]]$ ($[[UO B]]$)
must also hold. Note that the side condition of $[[UO B]]$
is essential. The COST algorithm will not be sound without
this condition.

\paragraph{Soundness and completeness of COST algorithm.}
We prove that the COST algorithm is sound and complete with
respect to the COST specifications.

\begin{lemma}[COST Soundness]
  $[[cost A B]]$ $\rightarrow$ A $\sqcap_s$ B.
\label{lemma:cost:sound}
\end{lemma}

\begin{lemma}[COST Completeness]
  A $\sqcap_s$ B $\rightarrow$ $[[cost A B]]$.
\label{lemma:cost:complete}
\end{lemma}


\section{Disjointness}

The disjointness for \name is the converse of COST.
Two types are disjoint if they do not share any ordinary subtype.
Whereas, the COST algorithm states the otherwise.
Therefore, the disjointness is simply the negation of COST and
is shown in \Cref{cost:disj:def}.

\begin{definition}[Disjointness Algorithm]
\label{cost:disj:def}
    A $*_a$ B $\Coloneqq$ $\neg$ ($[[cost A B]]$) \\
\end{definition}


\paragraph{Disjointness equivalence.}
We prove that the novel disjointness based on
COST is sound and complete with respect to standard disjointness
specifications. The disjointness specifications are shown again 
in \Cref{def:cost:disj} for readability.

\begin{definition}[$[[/\]]$-Disjointness]
\label{def:cost:disj}
  % A $*_s$ B $\Coloneqq$ $\forall$ C, \ $[[ordinary C]]$ \ $\Longrightarrow$ \ $\neg([[C <: A]]$ and $[[C <: B]]$).
  $[[A *s B]]$ $\Coloneqq$ $\nexists$ C, $[[ordinary C]]$ and $[[C <: A]]$ and $[[C <: B]]$.
\end{definition}

\begin{lemma}[Disjointness equivalence]
  $[[A *s B]]$ $\leftrightarrow$ $\neg$ (A $\sqcap_s$ B)
\label{lemma:cost:disj:sound}
\end{lemma}


% \begin{lemma}[Disjointness completeness]
%   $\forall$ $[[A]]$ $[[B]]$, $\neg$ (A $\sqcap_s$ B) $\rightarrow$ $[[A *s B]]$.
% \label{lemma:cost:disj:complete}
% \end{lemma}


\section{Typing, Operational Semantics, and Type-safety}

The typing and the operational semantics do not essentially
require revision for this chapter and are shown in \Cref{fig:cost:typ}.
Both of these relations are standard and have already been
explained.

\paragraph{Type-safety and determinism.}
The standard properties of the type-safety consisting
of type preservation and the progress hold in this system.
\Cref{lemma:cost:preservation} states type preservation 
and the \Cref{lemma:cost:progress} states progress.
We also show that the reduction is deterministic
(\Cref{lemma:cost:determinism}).

\begin{theorem}[Type Preservation]
\label{lemma:cost:preservation}
  If \ $[[G |- e : A]]$ and $[[e --> e']]$ then $[[G |- e' : A]]$.
\end{theorem}

\begin{theorem}[Progress]
\label{lemma:cost:progress}
If \ $[[G |- e : A]]$ then either $[[e]]$ is a value;
or $[[e]]$ can take a step to $[[e']]$.
\end{theorem}

\begin{theorem}[Determinism]
\label{lemma:cost:determinism}
  If \ $[[G |- e : A]]$ and \ $[[e --> e1]]$ and \ $[[e --> e2]]$ then $[[e1]]$ = $[[e2]]$.
\end{theorem}

\begin{figure}[t]
    \begin{small}
    \centering
    \drules[typ]{$[[G |- e : A]]$}{Typing}{int,null,var,app,sub,abs,and,switch}
    \drules[step]{$[[e --> e']]$}{Operational semantics}{appl,appr,beta,switch,switchl,switchr}
  \end{small}
  \caption{Typing for \name.}
  \label{fig:cost:typ}
\end{figure}