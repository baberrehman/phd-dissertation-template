%
\noindent
%

% A programming language is a tool to communicate with the computers.
% Several programming languages have been developed over the decades
% even prior to the invention of very first computer. Each new language
% comes with the new features.

Languages define the constructs and protocols of communication.
Programming languages, in particular, define the constructs that
assist in communicating with machines. With the growing complexity
and need of software applications, it has become a challenge to
design an expressive, yet simple programming language. The complexity
in a language design mostly occurs due to the interaction of
disparate features. This thesis is an effort in simplifying and formally
studying the design of intersection and union types in
programming languages.

Intersection and union types are advance and powerful features
available in many modern programming languages. Intersection types 
provide an interface to introduce an expression of multiple types.
Union types, on the other hand, provide an interface to express an
expression of variant types. Interaction of intersection and
union types is known to be non-trivial in theory.
This thesis examines the interaction of intersection and union types.
% We study a type-based switch expression for
% the elimination of union types called \name as a simple extension of
% simply typed lambda calculas. \name is then extended with more advance
% features such as intersection types, nominal types, subtyping distributivity,
% and disjoint polymorphism. \name and its extensions are proven to be
% type-sound and deterministic. Disjointness lies at the core of \name
% and its extension to prove determinism. Disjointness ensures that all
% the branches of a switch expression are non-overlapping.
Our study starts with a deterministic type-based switch expression
for the elimination of union types. Disjointness plays an integral
role in keeping the calculus deterministic, which ensures that no
two branches of a type-based switch expression overlap. Thus the
scrutinee falls in a maximum of one branch. The resulting calculus
is called \name. We further extend \name with powerful and advance
features of intersection types, subtyping distributivity, nominal
types, and polymorphism. Moreover, we study \name with the
merge operator. An extension with the merge operator poses
novel challenges in determinism.

Intersection and union types are well-known dual features.
We also examine 
the duality of intersection and union types formally in this thesis.
The duality unifies some of the subtyping rules and reduces the theoretical complexity
of a system with dual features.
The benefits include the reduction in number of subtyping rules and simplified proofs
for certain theorems such as subtyping transitivity. 
All of the metatheory of this thesis has been
formalized in the Coq theorem prover.

\vspace{1.5\baselineskip}

\noindent\makebox[\linewidth]{\rule{0.7\textwidth}{0.4pt}}

\begin{center}
\textbf{An abstract of 310 words}
\end{center}

\newpage

\begin{flushright}
  \null\vspace{\stretch{1}}
  \textit{To my late father...}
  \vspace{\stretch{2}}\null
\end{flushright}
