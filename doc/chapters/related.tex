
%%%%%%%%%%%%%%%%%%%%%%%%%%%%%%%%%%%%%%%%%%%%%%%%%%%%%%%%%%%%%%%%%%%%%%%%
\chapter{Related Work}
\label{chap:related}
%%%%%%%%%%%%%%%%%%%%%%%%%%%%%%%%%%%%%%%%%%%%%%%%%%%%%%%%%%%%%%%%%%%%%%%%


%%%%%%%%%%%%%%%%%%%%%%%%%%%%%%%%%%%%%%%%%%%%%%%%%%%%%
%%%%%%%%%%%%%%%%%%%%%%%%%%%%%%%%%%%%%%%%%%%%%%%%%%%%%
%%
%% Related work for Disjoint Switches
%%
%%%%%%%%%%%%%%%%%%%%%%%%%%%%%%%%%%%%%%%%%%%%%%%%%%%%%
%%%%%%%%%%%%%%%%%%%%%%%%%%%%%%%%%%%%%%%%%%%%%%%%%%%%%


\begin{comment}
\begin{itemize}
	\item {Union Types    (discussing work on union types and elimination constructs for union types) <-- Need to coordinate with Ningning/Snow so that there’s not much overlap there. If the Overview discusses some of these in detail, the RW does not need to have much text in it.}
	\item {Overloading (more general work on overloading, including type classes, approaches to overloading based on intersection types, etc…)}
	\item{Disjointness and Disjoint Intersection Types}
\end{itemize}
\end{comment}


%---------------------------------------------------%%

%%%%%%%%%%%%%%%%%%%%% Union Types  %%%%%%%%%%%%%%%%%%%

%---------------------------------------------------%%


\section{Union Types}
\begin{comment}
Set-theoretic unions have sound theory and extensively studied in
mathematics. Set-theoretic unions correspond to union types or
disjoint union types in programming languages. Disjoint union types
are also called sum types or variants.  Constructors are explicitly
labeled in disjoint union types and expressions are manipulated using
corresponding labels. Few other interesting calculi (and this paper)
do not use labels and provide type-based union elimination.
%Developing
%a sound and deterministic union elimination construct for type-based
%union elimination have been a challenge in research community.
\end{comment}
Union types were first introduced by \cite{macqueen1984ideal}.
They proposed a typing rule that eliminates unions implicitly.
The rule breaks type preservation under
the conventional reduction strategy of the lambda calculus.
\cite{barbanera1995intersection} solved the problem by reducing all
copies of the same redex in parallel.
\cite{dunfield2003type} and \cite{dunfield2014elaborating} took another approach to support
mutable references. They restricted the elimination typing rule to only allow a single
occurrence of a subterm with a union type when typing an expression.
\cite{pierce1991programming} proposed a novel single-branch case construct
for unions. As pointed by Dunfield and Pfenning, compared to the single
occurrence approach, the
only effect of Pierce's approach
is to make elimination explicit.
\begin{comment}
Moreover, while \cite{dunfield2014elaborating} shows that
subtyping is not necessary for elaboration, it is not obvious how to generalize
elaboration to support subtyping relations such as
\lstinline{Student <: Person} without using the subtyping rule. If the
elaboration were generalized further to support such a subtyping relation, then
a student with type \lstinline{Student | Person} can also be tagged
non-deterministically.
\end{comment}


% \cite{barbanera1995intersection} proposed
% two approaches for type preservation in a calculus with intersection and union
% types.
%However, their union elimination is same as last two and follow
%unrestricted union elimination.

\begin{comment}
\cite{freeman1991refinement} studied union types along with intersection types
in setting of refinement types for ML. Main focus of their work is to infer
more precise types of expressions, which they call
refinement types. Their work is targeted to contribute for the types
in ML and not for the expressions. Therefore, they did not define
expressions and dynamic semantics. On the contrary, our work provides
a complete calculus with type sound dynamic semantics.
\bruno{and ... Do they have an elimination construct for unions?
  If so how is that related to our work?}.
\baber{I added few more lines in Freeman citation.}
\end{comment}
Union types and elimination constructs based on types are
widely used in the context of XML processing languages~\citep{hosoya2003xduce,benzaken2003cduce},
and have inspired proposals for object oriented languages \citep{igarashi2006union}.
Generally speaking, the elimination constructs in those
languages offer a first-match semantics,
where cases can overlap and reordering the cases may change the semantics of the program.
This is in contrast to our approach.
Union types have also been studied in the context of
XDuce programming language \citep{hosoya2003xduce}. XDuce employs
regular expression types. Pattern matching can be on
expressions and types in XDuce.  Expressions are considered as special
cases of types.  CDuce
\citep{benzaken2003cduce} is an extension of XDuce.
Work on the more foundational aspects
of CDuce, and in particular on \emph{semantic subtyping}~\citep{frisch2002semantic}
and set-theoretic types,
also employs a form of first-match semantics elimination construct, though in a different form.
In particular, work by ~\cite{castagna2005gentle} and \cite{castagna2017gradual}
propose a conditional construct that can test whether a value matches a type.
If it matches then the first branch is executed and the type for the value is refined.
Otherwise, the second branch is executed and the type of the value is refined to be
the negation of the type (expressing that the value does not have such type).
% \baber{I am not sure if we should refer to following related work suggested by R4.
% These papers are all about the design of algorithmic subtyping for OO languages
% with semantic subtyping. Union types are natural to have in set-theoretic systems
% such as semantic subtyping.}
Union types are also studied in the context of
semantic subtyping and object-oriented calculi
\citep{ancona2016semantic,ancona2014sound,dardha2013semantic} which
focus on designing subtyping algorithms to employ semantic subtyping in OOP.
In contrast, we study a deterministic and type-safe
switch construct for union elimination.
%along with union types.

\begin{comment}
\cite{fallside2001xml} studied union
types in markup language, but with a restriction of disjoint top-level
labels which is different from our work on
disjointness on types.\bruno{In other words, for the last work, the unions
are labelled (i.e. the cases are not based on types)?}

\cite{frisch2002semantic} studied union types and intersection types
with semantic subtyping and provide the theoretical basis for CDuce programming language.
However, their case expression (they call it \emph{match}) uses explicit
tags to select a particular branch. Whereas, we propose dynamic type-based
case construct.
\cite{castagna2005gentle} extended this work with type-based case construct
in the context of XDuce programming language. Main motivation of their
work is to elaborate the significance of semantics subtyping.
However, their underlying technique to select the branch in case construct
relies on negation types. We, on the other hand, use disjointness
and static type of the expression. \baber{they also use an extra variable.}
%\cite{frisch2008semantic}.
%\baber{more text.}
\cite{castagna2017gradual} studied
set-theoretic union types, intersection types and negation types
in gradual typing setting. They also propose a dynamic type-based
construct for set-theoretic union elimination which is closely
related to our work.
However, their underlying technique to select branches in case
expression is different from ours.
In particular, negation types play fundamental role in their
typing construct for the case expressions.
Whereas, our work relies on disjointness along with a notion of
static types to to select a particular branch.
They also do not provide dynamic semantics for their source language.
Whereas, we provide a type sound dynamic semantics for \name.
\baber{no variables in their case construct?}

\baber{below text will be removed for Castagna's work.}
\bruno{Castagna did alot of work
  on union types before this one, so I'm not sure why we are mentioning only this work
  of Castagna, instead of talking about previous work where the ideas were introduced.}
\baber{Castagna's work includes CDuce, which is mentioned. I added semantic subtyping
as well now. Should we add more Castagna's work? I think his above mentioned work
is relevant to our work.}
They give a dynamic type-based cast for union elimination. But they do not
have disjointness constraint on case branches and they check the type
only for first case branch\bruno{Not a very good explanation: you get negated types for the branches}.
\end{comment}

\cite{muehlboeck2018empowering} give a general framework for subtyping
with intersection and union types. They illustrate the significance of
their framework using the Ceylon programming language.
The main objective of their work is to define a generic framework for
deriving subtyping algorithms for
intersection and union types in the presence of various distributive subtyping rules.
For instance, their framework could be useful to derive an algorithmic
formulation for the subtyping relation presented in Figure~\ref{fig:dist:dec:subtyping}.
They also briefly cover disjointness in their work. As part of their framework, they
can also check disjointness given some disjointness axioms. For instance,
for \name, such axioms could be similar to \rref{ad-btmr} or \rref{ad-intl}
in Figure~\ref{fig:union:disj-typ}.
However, they do not have a formal
specification of disjointness. Instead they assume that some sound specification
exists and that the axioms respect such specification.
If some unsound axioms are given to their framework (say $[[Int * Int]]$) this
would lead to a problematic algorithm for checking disjointness.
In our work we provide specifications for disjointness together
with sound and complete algorithmic formulations.
In addition, unlike us,
they do not study the semantics of disjoint switch expressions.
%although they mention some Ceylon examples using such expressions.

\paragraph*{Occurrence Typing.}
Occurrence typing or flow typing \citep{tobin2008design} specializes or refines
the type of variable in a type test. An example of occurrence typing is:

\begin{lstlisting}
	Integer occurrence (Integer | String val) {
	  if (val is Integer) { return val+1; }
	  else                { return toInt(val)+2; }
	}
\end{lstlisting}

\noindent In such code, \lstinline{val} initially has type \lstinline{$[[Int \/ String]]$}.
The conditional checks if the \lstinline{val} is of type \lstinline{$[[Int]]$}.
If the condition succeeds, it is safe to assume that \lstinline{val} is of type \lstinline{$[[Int]]$},
and the type of \lstinline{val} is refined in the branch to be \lstinline{$[[Int]]$}.
Otherwise, it is safe to assume that \lstinline{val} is of type \lstinline{$[[String]]$}, in the
other branch (and the type is refined accordingly).
The motivation to study occurrence typing was to introduce typing in dynamically
typed languages.
Occurrence typing was further studied by \cite{tobin2010logical},
which resulted into the development of Typed Racket.
Variants of occurrence typing are nowadays employed in mainstream languages
such as TypeScript, Ceylon or Flow.
% \cite{castagna2019revisiting} proposed a more general formulation of
% occurrence typing.
\cite{castagna2019revisiting} extended occurrence typing to refine the type of
generic expressions, not just variables. They also studied the combination
with gradual typing. Recently, \cite{castagna2022type} show that the
classical union and intersection types along with a type-based
switch construct encompasses occurrence typing.
Occurrence typing in a conditional construct,
such as the above, provides an alternative
means to eliminate union types using a first-match semantics. That is the
order of the type tests determines the priority.
%that is given to the various types being tested.

\paragraph*{Nullable Types.}
% Generally speaking, \lstinline{Null} type
% has only one value i.e \lstinline{null}. \lstinline{Null} type distinguishes itself from $[[Bot]]$
% type in a sense that $[[Bot]]$ has no inhabitants at all while \lstinline{Null}
% has onle one inhabitant i.e \lstinline{null}.
% In that view, $[[Bot]]$ type is the empty
% set while \lstinline{Null} type is a singleton set.
% Value of \lstinline{Null} type can be used as
% value of various other types such as \lstinline{String} in Scala (except Scala 3).
Nullable types are types which may have the \lstinline{null} value.
%Implicit nulls add tricky challenges and runtime errors in a program
%if the care is not taken.
Recently, \cite{nieto20nulls} proposed an
approach with explicit nulls in Scala using union types. The Ceylon
language has implemented a similar approach for a few years now.
However our's and Ceylon's approaches are based on disjoint switches to test
for nullability, while \cite{nieto20nulls}'s approach is based
on a simplified form of occurrence typing.

%While approach in ~\cite{nieto20nulls} detects \lstinline{null} value deference at
%compile time by reifying subtyping hirearchy in Scala.

Various approaches have been proposed to deal with nullability
such as \lstinline{T?} in Kotlin~\citep{kotlin},
Swift~\citep{swift} and Flow~\citep{chaudhuri2017fast}.
The Checker Framework \citep{papi2008practical}
is another line of related work to detect null pointer deferences in Java programs.
\cite{banerjee2019nullaway} proposes an approach to explicitly associate
nullable and non-nullable properties with expressions in Java.
However, differently from our work,
in those approaches nullable types are not encoded with union types.
\cite{matchtypes2022} study a notion of match types for type level
programming. They also employ a notion of disjointness in match types
and can encode nullable types. However, they provide
match types at the type level and do not use them for union elimination.
Furthermore, they do not study intersection and union types formally.
In contrast, we provide a term level switch construct for union elimination.

%\baber{They have both term level and type level match construct.}

%All these approaches diverge from our work. We formalize a type-safe
%and deterministic switch construct based upon disjointness with union and
%intersection types.
%While these approaches deal with \lstinline{null} pointer deferences by
%exploiting various techniques such as optional types, explicit tags or
%program analysis.


%---------------------------------------------------%%

%%%%%%%%%%%%% Disjoint Intersection Types  %%%%%%%%%%%

%---------------------------------------------------%%


\section{Disjoint Intersection Types}
%\cite{pottinger1980type} and \cite{coppo1981functional} were the first to study
%intersection types in the research
%literature. Forsythe~\cite{reynolds1988preliminary} is the first
%programming language to have intersection types, but it did not
%have union types.
Disjoint intersection types were first
studied by \cite{oliveira2016disjoint} in the $\lambda_{i}$ calculus
to give a coherent calculus for intersection types with a merge
operator. The notion of disjointness used in \name, discussed in \Cref{sec:union},
is inspired by the notion of disjointness of $\lambda_{i}$. In essence,
disjointness in \name is the dual notion: while in $\lambda_{i}$ two types
are disjoint if they only have \emph{top-like} supertypes, in \name two types
are disjoint if they only have \emph{bottom-like} subtypes.
\emph{Disjoint polymorphism}~\citep{alpuimdisjoint} has been studied for
calculi with disjoint intersection types.

None of calculi with disjoint intersection types
\citep{oliveira2016disjoint,bi_et_al:LIPIcs:2018:9227,alpuimdisjoint,bi:fiplus} in the literature
includes union types. One interesting discovery of our work is that the
presence of both intersections and unions in a calculus can affect disjointness.
In particular, as we have seen in Section~\ref{sec:inter}, adding intersection types
required us to change disjointness. The notion of disjointness that was
derived from $\lambda_{i}$ stops working in the presence of intersection types.
Interestingly, a similar issue happens when union types are added to
a calculus with disjoint intersection types. If disjointness of two types \lstinline{A}
and \lstinline{B} is defined to be that such types can only have top-like types,
then adding union types immediately breaks such definition.
For example, the types \lstinline{$[[Int]]$} and \lstinline{$[[Bool]]$} are disjoint but, with union
types, \lstinline{$[[Int \/ Bool]]$} is a common supertype that is not top-like.
We conjecture that, to add union types to disjoint intersection types,
we can use the following definition of disjointness:
\vspace{-1mm}
\begin{definition}
\label{def:related:disj}
  %A $*_s$ B $\Coloneqq$ $\forall$ $[[Cord]]$, $\neg$ \ ($[[A <: Cord]]$ and $[[B <: Cord]]$).
  $[[A *s B]]$ $\Coloneqq$ $\nexists$ $[[Cord]]$, $[[A <: Cord]]$ and $[[B <: Cord]]$.
\end{definition}
\vspace{-1mm}
\noindent which is, in essence, the dual notion of the definition presented in
Section~\ref{sec:inter}. Under this definition \lstinline{$[[Int]]$} and \lstinline{$[[Bool]]$} would
be disjoint since we cannot find a common ordinary supertype (and \lstinline{$[[Int \/ Bool]]$}
is a supertype, but it is not ordinary). Furthermore, there should be a
dual notion to LOS, capturing the greatest ordinary supertypes. Moreover,
if a calculus includes both disjoint switches and a merge operator,
then the two notions of disjointness must coexist in the calculus.
This will be an interesting path of exploration for future work.


\section{Overloading and Dynamic Dispatch}

\paragraph*{Overloading.}
Union and intersection types also provide a form of function overloading or ad-hoc
polymorphism using the switch and type-based case analysis. A programmer
may define the argument type to be a union type. By
using type-based case analysis, it is possible to execute different code
for each specific type of input.  Intersection types have also been
studied for function overloading. For example, a function with type
\lstinline{$[[Int -> Int /\ Bool -> Bool]]$} can take input values either of type
\lstinline{$[[Int]]$} or \lstinline{$[[Bool]]$}.  In such case,
it returns either \lstinline{$[[Int]]$} or \lstinline{$[[Bool]]$}
depending upon the input type.  Function overloading
\citep{castagna1995calculus, cardelli1985understanding, wadler1989make}
has been studied in detail in the literature.
%by \cite{castagna1995calculus},
%\cite{cardelli1985understanding}, \cite{stuckey2005theory} among others.
\cite{wadler1989make} studied type classes as an alternative way
to provide overloading based on parametric polymorphism.
Recently, \cite{rioux2023bowtie} studied function overloading with
intersection types, merge operator, and the union types
in a calculus called $F_{\bowtie}$.
Their calculus is type-safe and deterministic.
However, the merge operator in their calculus is restricted
to functions. Moreover, the $F_{\bowtie}$ calculus
is proposed after \cite{rehman_et_al:LIPIcs.ECOOP.2022.25}.


\paragraph{Dynamic dispatch.}
A straightforward definition of dynamic dispatch
\citep{castagna1995calculus,bourdoncle1997type,castagna2014polymorphic,clifton2006multijava}
is the \emph{runtime function overloading}.
In contrast to the static overloading, a specific
function implementation is selected based on the the dynamic or
runtime type of the one (usually left) or more argument(s) in dynamic dispatch.
This division categorizes the dynamic dispatch into single and multiple
dispatch respectively. Single dispatch comes naturally with most of the
OOP languages such as C++ \citep{stroustrup1986overview} and
Smalltalk \citep{goldberg1983smalltalk}.
Multiple dispatch
\citep{castagna1995calculus,bourdoncle1997type,castagna2014polymorphic},
on the other hand, requires special care.
Normally, types overlap in multiple dispatch and
the language implementation selects the best matching function
implementation based on the argument type such as in
Julia \citep{zappa2018julia}.
Recently, \cite{park2019polymorphic} study symmetric dynamic dispatch with
multiple inheritance, parametric polymorphism, and variance.
% They proposed a type-sound calculus.
% Their study consists of a static as well as the dynamic semantics.
However, this line of work differs from ours in a way that we do not allow
overlapping types in \name in the alternative branches of a switch expression.
Thus \name provides a less expressive but deterministic variant of dynamic dispatch.
% In particular, the types in two different implementations of a function
% cannot overlap.


%---------------------------------------------------------------------------%%



%---------------------------------------------------%%

%%%%%%%%%%%%%%%%%%% Duality in Logic  %%%%%%%%%%%%%%%%

%---------------------------------------------------%%


\section{Duality in Logic and Programming Language Theory}

Apart from informally observing duality of type system features, as far as we known,
formally exploiting duality in subtyping relations has not been investigated in the past.
However there is plenty of work on uses of duality in programming language theory.
Furthermore there is related work on type
systems that exploit various generalizations for added expressive
power or economy in metatheory and implementation. We discuss these next.

In type theory~\citep{Andrews86} and/or category theory~\citep{MacLane:cwm,BirddeMoor96:Algebra}
duality occurs in various forms. For instance,
the duality between sum and product types is well-known in both type and
category theory. Properties about such types often explicitly acknowledge
duality. Many properties about sum types are presented as 
dual properties of corresponding properties on product types and vice-versa. Our Lemma \ref{lemma:ddboth-inv}
is an example of a property that applies to both union and/or intersection types.
In this property duality is not only acknowledged, but directly exploited in the lemma
itself to provide a generalized property that can be specialized
to one construct and its dual.
Various other dualities between constructs are known and exploited in various
ways in type and/or category theory.
For example, existential and universal quantification can be captured by an
encoding by one through the other. The type $\exists \alpha.~A$ can
be encoded as $\forall \beta. (\forall \alpha.~A \to \beta) \to \beta$, which
requires a kind of CPS translation~\citep{DaFi1992} of the corresponding terms. Similar
encodings exist for sums and products.

In the field of \emph{proof-theoretic semantics}~\citep{gentzen:untersuchungen}
and in \emph{natural deduction}
the concept of \emph{harmony} is used to describe introduction and
elimination rules that are in some sense dual. For instance, the usual
rules for introduction and elimination of conjunction are in perfect harmony.
The inversion principles by \cite{Prawitz1979ProofsAT}
are a general procedure to associate to any
arbitrary collection of introduction rules a specific collection of
elimination rules. The elimination rules are in harmony with the given collection of
introduction rules. Prawitz inversion principles attempt to capture harmony
in a more precise way, directly expressing it formally.
Therefore inversion principles have similar considerations to \nameduo in terms
of expressing some form of duality directly in a formalism. However
inversion principles focus on introduction and/or elimination rules,
while \nameduo is focused on subtyping. Nevertheless in future work we
are interested in exploiting the use of duality in the typing relation more.
We believe that the notion of harmony and inversion principles could be
quite helpful in such work.

\emph{Double-line rules}~\citep{dosen1989} are deduction rules that
can be read both from top to bottom (as usual) and
also from bottom to top. In other words they express
two standard (dual) deduction rules in a single double-line rule.
Like \nameduo, double-line rules aim at expressing a form of duality
in a single rule. Unlike \nameduo, double-line rules are concerned
with (dual) rules where the premises and conclusions of one rule
become the conclusions and premises of the other rule, respectively.

\cite{bernardi2014duality} explain duality relations in the
context in \emph{session types}.  Binary session types have two
endpoints connected through one communication channel.
In session types, connected endpoints should have a
dual relation in their session types. The duality relation in session
types is related to types and may have various
interpretations. In contrast \nameduo is about subtyping (or supertyping).

The duality between data and codata is well-known in programming
language theory~\citep{BirddeMoor96:Algebra}. Data types and codata types are duals in the sense
that data types are defined in terms of constructors while codata
types are defined in terms of destructors.  More recently, such
duality has been exploited in language design
\citep{ostermann2018dualizing,binder2019decomposition} to provide an
automatic way to switch between programs defined on datatypes and equivalent
programs defined on codata types. 
The use of duality in this line of work is quite different from ours.


%---------------------------------------------------%%

%%%%%%%%%%%%%%%%%%% Generalizations  %%%%%%%%%%%%%%%%%

%---------------------------------------------------%%


\section{Generalizations in Type Systems and Type Theory}

\emph{Pure type systems} (PTSs)
\citep{van1993checking,mckinna1993pure,adams2006pure,jutting1993typing,severi1994pure,zwanenburg1999pure}
capture a generalization of various type systems ($F,
F^{\omega}, {\lambda}$P). Typing rules of multiple type systems are
expressed in pure type systems via parameterization.
PTSs are parameterized by three sets: a set of sorts; a set of axioms;
and a set of rules. Concrete type systems (such as System $F$),
are recovered with concrete instantiations of those sets.
Pure type systems with subtyping
\citep{zwanenburg1999pure} are a variant of pure type systems
that captures a family of type systems with subtyping. This
variant captures only the upper bounds. It does not provide subtyping
generalization with both upper and the lower bounded quantification
like our \nameduo generalizations of $F_{<:}$.
Pure subtype systems \citep{hutchins2010pure}
is a family of calculi based on subtyping only (and without a typing
relation). This system eliminates the need of typing and presents an
alternative to typing using subtyping only.
Pure subtype systems support upper bounded quantification, but no lower
bounded quantification.

\paragraph{Modal Type Theory.}
Modal type theory~\citep{nanevski2008contextual}
is an extension of type theory which provides type
rules using modalities. Modal type
theory can represent a proposition as
types which may be proved based upon the deduction rules in a given
context. Modal type theory also employs modes, for instance \emph{possibility} and 
\emph{necessity}~\citep{simpson1994proof, nanevski2008contextual}.
There are many type systems that use modes to generalize
typing relations. One can view \nameduo as a simple instance of a
relation with a mode. In \nameduo the mode is either subtyping or
supertyping.

\paragraph{Bi-directional type checking.}
Bi-directional type checking \citep{local,dunfield2019bidirectional}
also employs a mode, but in the typing relation
instead. Bi-directional type checking is a common technique, used in implementations
of programming languages, that can eliminate redundant type
annotations.  Bi-directional type-checking is also employed is
several type systems, especially those where full type inference is
undecidable~\citep{local,dunfield2013complete}.  In such cases only
partial inference methods are feasible in practice, which means that
some type annotations are necessary. Bi-directional type checking is
useful in such cases, allowing the type information to be easily
propagated without requiring further (redundant) annotations.  The
modes in bi-directional type-checking are checking or synthesis.
Checking checks a given term against a given type, whereas the
synthesis infers the type based upon the available information in the
context. 

\paragraph{Unified Subtyping.}
Unified subtyping \citep{yang2017unifying} is a technique that can be used
in dependently typed systems supporting unified syntax to model typing
and subtyping in a single relation. The single unified subtyping relation
generalizes both typing and subtyping. Like \nameduo, unified subtyping
can also help reducing language metatheory and duplication. However
unified subtyping is orthogonal to \nameduo and does not exploit duality
of features. We believe that both techniques can complement each other.

\begin{comment}
Lower-bounded quantification is not really new.  Igarashi and Viroli
>   have pointed out correspondence between (use-site) variance and
>   existential types and, in order to capture contravariance, they
>   introduced lower-bounded existential types.
> 
>   Atsushi Igarashi, Mirko Viroli: On Variance-Based Subtyping for
>   Parametric Types. ECOOP 2002: 441-469
\end{comment}

\paragraph{Bounded quantification and generalizations.}
System $F_{<:}$ \citep{cardelli1994extension} 
is extensively studied due to its feature of bounded
quantification. F-bounded
quantification \citep{canning1989f} is a generalization of bounded
quantification to handle recursive types.
Although we are not aware of an extension of $F_{<:}$ with lower
bounded quantification, such notion has appeared before
in some calculi. For instance, \cite{igarashi:2002:variance}
have pointed out correspondence between use-site variance and
existential types and, in order to capture contravariance, they
introduced lower-bounded existential types.

One generalization of $F_{<:}$ is studied by \cite{amin2017type}, which formalizes
\emph{type bounds} in Scala.
Type bounds is an interesting feature in Scala as elaborated by the following code (code extended from \cref{sec:overview:4}):

\begin{lstlisting}[language=Scala]
	class TypeBoundsCollection[S >: GraduateStudent <: Student](obj: S) {
	  def student: S = obj
	}
\end{lstlisting}

\noindent While in our variants of $F_{<:}$ we support either lower bounded
quantification or upper bounded quantification (but not both at once), Scala's
type bounds allow both upper and lower bounds at once. This is clearly more
expressive than what we have, but it comes with its own problems.
Formalisms with Scala-like type bounds often need to include a transitivity
axiom (and thus are non-algorithmic) and they have to deal with the bad bounds
problem. In contrast our simpler extension of type bounds is comparable
in complexity to $F_{<:}$'s upper bounded quantification, and there is a set of
algorithmic subtyping rules without a built-in transitivity axiom.

% \paragraph{Intersection and Union Types}
% Intersection and union types
% \citep{barbanera1995intersection,dunfield2014elaborating,muehlboeck2018empowering,pierce2002programming}
% are getting significant attention in recent years, and are used in
% several modern programming languages (including Scala, Flow or
% TypeScript). \cite{reynolds1988preliminary} was the first to
% promote the use of intersection types in programming languages. Later
% on, \cite{pierce2002programming} studied intersection types,
% union types and polymorphism combined in a typed $[[\]]$-calculus.
% Recently, \cite{muehlboeck2018empowering} presented a generalized
% formulation of calculi with union and intersection types. They
% demonstrated it with the help of Ceylon programming language \citep{king2013ceylon}.
% %Refinement intersection types.
% \cite{dunfield2014elaborating} presented an expressive calculus with a merge
% operator and unrestricted intersection types with union types.
% We exploit the duality of union and intersection types to illustrate \nameduo.
% Our \nameduo calculi manages to capture the six common rules for unions and intersections
% using three rules only (plus the duality rule), which provides a simple 
% illustrative example of the use of duality.


\paragraph*{Duality in Subtyping of Intersection and Union Types.}
We exploit the duality of union and intersection 
types to illustrate \nameduo.
Our \nameduo calculi manages to capture the six common 
rules for unions and intersections
using three rules only (plus the duality rule), which provides a simple 
illustrative example of the use of duality.
None of the calculi with intersection and union types discussed so far
study the duality of subtyping formally as we do.

%---------------------------------------------------------------------------%%


%%% Local Variables:
%%% mode: latex
%%% TeX-master: "../Thesis"
%%% org-ref-default-bibliography: "../Thesis.bib"
%%% End:
